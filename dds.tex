\documentclass[a4paper,12pt]{article}

% Hinrich
%\usepackage{helvet}
\usepackage{natbib}
\usepackage{bibentry}

\def\secsref#1#2{Sections~\ref{sec:#1}--\ref{sec:#2}}
\def\figref#1{Figure~\ref{fig:#1}}
\def\figlabel#1{\label{fig:#1}\label{p:#1}}
\def\Tabref#1{Table~\ref{tab:#1}}
\def\tabref#1{Table~\ref{tab:#1}}
\def\tabsref#1#2{Tables~\ref{tab:#1}-\ref{tab:#2}}
\def\figsref#1#2{Figures~\ref{fig:#1}-\ref{fig:#2}}
\def\figrefand#1#2{Figures~\ref{fig:#1} and \ref{fig:#2}}
\def\tablabel#1{\label{tab:#1}\label{p:#1}}
\def\eqref#1{Eq.~\ref{eqn:#1}}
\def\eqlabel#1{\label{eqn:#1}}
\def\Secref#1{Abschnitt~\ref{sec:#1}}
\def\seclabel#1{\label{sec:#1}\label{p:#1}}
\def\secref#1{Section~\ref{sec:#1}}

% Marco
\usepackage{linguex}


\title{The Working Group Formerly Known as Dynamic Distributional
  Semantics at MIC}

\begin{document}

\maketitle

\section{Motivation}
\label{sec:motivation}

Consider:

\ex.\label{ex1} \a. The doctors are old.
\b. The cucumbers are old.

We expect a good Distributional Semantic (DS) model to build sensible
representations of these two sentences. For example, the model should
be able to capture how the subject noun disambiguates the predicative
adjective, so that \Last[b] is very similar to \emph{``The cucumbers
  are rotten''}, but \Last[a] is most definitely not a paraphrase of
\emph{``The doctors are rotten.''} Consider next:

\ex. \label{ex:discourse-context}
\a. \a.I saw the doctors\ldots 
    \b. I saw the cucumbers\ldots
\z.
\b.\label{ex:they-are-old} They are old.

The interpretation of \Last[b] depends on whether it is a continuation
of \Last[a-i] or \Last[a-ii]. It is not clear that DS would be able to
construct an appropriate representation of \Last[b], e.g., one that is
closer to \emph{``They are rotten''} only if \emph{they} refers to the
cucumbers.

Making sense of \Last[b] crucially depends on a notion of
\emph{reference}.  Correctly interpreting \emph{they} requires
referring to some entities that are currently in the universe of
discourse of the unfolding conversation, e.g., a specific set of
cucumbers. Reference to discourse entities is actually also needed to
make full sense of the sentences in \LLast (\emph{which} doctors or
cucumbers are we referring to?), but, there, at least predicate
disambiguation (assigning the right sense to \emph{old}) can be
performed by considering \emph{generic} properties of doctors and
cucumbers (the latter are more likely to go bad than the former), that
do not require considering the specific subsets of entities we are
talking about.

DS models, by extracting word meanings from large amounts of text, can
be very effective at capturing generic properties, but, for the same
reason (word representations are averages across many contexts), they
are not well-equipped to handle phenomena depending on specific
reference (see Section \ref{sec:examples} for other phenomena of this
sort).

Still, the generic knowledge encoded in DS might provide evidence that
can help to solve specific reference ambiguities. Consider:

\ex. \a. The professors finally found the cucumbers\ldots
\b. They were rotten.

It is easy to think of approaches that could harness DS to decide that
\emph{they} must refer to cucumbers. For this specific example,
something as simple as measuring distributional similarity of
\emph{professor/rotten} vs.~\emph{cucumber/rotten} might suffice. Of
course, most real-life cases of coreference resolution require a much
more sophisticated exploitation of DS-based cues (and in many cases DS
will not suffice), but at least the issues at hand are clear: What
kind of DS models are best for the task? How should DS-produced scores
be integrated in a larger coreference system? Etc.

We would like to use the opportunity offered by the symposium to
instead focus most directly on the relation between DS and reference
from the point of view of a general theory of DS. That is, we are not
going to discuss how to use DS evidence as a component in other
systems that handle reference, but rather how to handle
reference-related phenomena \emph{within} DS.

\section{Examples}
\label{sec:examples}

To make the working group discussion concrete, we propose a number of
examples we would like you to think of, and we invite you to add more
to the list.

\paragraph{Plain pronouns,} where disambiguation really depends on
context and not on generic conceptual properties.

\ex. \a. The teacher entered the classroom\ldots 
\b. He looked scary.

How do we build a distributional representation of \Last[b] that
captures, e.g., the fact that it entails: \emph{``The teacher looked
  scary''}?

\paragraph{Definiteness.}

How do we represent the difference between:

\ex. \a. The teacher entered the classroom.
\b. A teacher entered the classroom.

How do the representations of these sentences change if, instead of
being uttered out of context, they are produced while speaking of the
classroom where Mr.~Smith teaches?

\paragraph{Proper nouns.}

Assuming Mr. Smith (the teacher) is called John, how do we capture the
relation between:

\ex. \a. A teacher entered the classroom.
\b. The teacher entered the classroom.
\c. John entered the classroom.

How do we capture the fact that, given that we know which John we are
talking about, \emph{John} in \Last[c] carries presuppositions that
are more like those of \emph{``the teacher''} in \Last[b] than those
of \emph{``a teacher''} in \Last[a]?

\paragraph{Referentially identical: same or different
  vectors? \emph{Sinn} or \emph{Bedeutung?}}

The following example is from Gemma Boleda's ongoing Marie Curie
project on DS and reference:

\ex. \a. The Beatles broke up in 1970.
\b. The best-selling band of the sixties broke up in 1970.



%\section{Topic}
% (Representation of) reference in distributional semantics

% Not a topic: how to use DS (as a component in other systems)
% to deal with reference

\section{Questions}
\label{sec:questions}

We would like you to think about the previous examples (and of course
more that you come up with) while pondering the following questions:

\begin{enumerate}
\item How do we \textbf{represent sentences} such as those in these
  examples in DS?
\item How do we \textbf{represent the broader discourse context} and
  the entities it defines in DS?
\item How do we \textbf{build such representations compositionally}?
\item How do we \textbf{learn a reference-aware DS model} from the
  data? From \textbf{which} data?
\end{enumerate}

Of course, we realize that a reasonable answer to these questions is
that DS was never meant to handle reference, so that they should not
even be asked. This is fair game for what we should discuss in the
working group.


\section{Subtopics and questions}

\textbf{MB: I would not attempt to review the literature in this
  document.}

\paragraph{Commonsense-based coreference.}
coreference that can be solved based on general commonsense
knowledge, of the sort that distributional semantics should
be good at capturing.
(Winograd schemas)

\paragraph{DS models of slots.} Ji and Eisenstein paper.

\paragraph{Maybe we do not need to do anything,
  vectors/embeddings suffice?} Aurelie Herbelot paper.

\paragraph{DS was never meant to be a theory of
  everything. Is reference one of the phenomena that we
  should not use DS to explain?}

\paragraph{Reference in the sense of the Pado/Baroni EMNLP
  paper.} But this is actually not about the reference that
we define as the topic?


\section{Stuff that will be cut, but that I moved here, so
  it is preserved}

\bibliography{buecher}
\bibliographystyle{apalike}

\end{document}


