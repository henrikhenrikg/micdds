\documentclass[a4paper,12pt]{article}

%\usepackage{helvet}
\usepackage{natbib}
\usepackage{bibentry}

\def\secsref#1#2{Sections~\ref{sec:#1}--\ref{sec:#2}}
\def\figref#1{Figure~\ref{fig:#1}}
\def\figlabel#1{\label{fig:#1}\label{p:#1}}
\def\Tabref#1{Table~\ref{tab:#1}}
\def\tabref#1{Table~\ref{tab:#1}}
\def\tabsref#1#2{Tables~\ref{tab:#1}-\ref{tab:#2}}
\def\figsref#1#2{Figures~\ref{fig:#1}-\ref{fig:#2}}
\def\figrefand#1#2{Figures~\ref{fig:#1} and \ref{fig:#2}}
\def\tablabel#1{\label{tab:#1}\label{p:#1}}
\def\eqref#1{Eq.~\ref{eqn:#1}}
\def\eqlabel#1{\label{eqn:#1}}
\def\Secref#1{Abschnitt~\ref{sec:#1}}
\def\seclabel#1{\label{sec:#1}\label{p:#1}}
\def\secref#1{Section~\ref{sec:#1}}

\title{The DDS WG at MIC}

\begin{document}

\maketitle

\section{Introduction}
Then, we could circulate them, explaining that, also based
on the feedback
from Bonnie and Alessandro, we identified reference in DS as
a topic of
common interest, but we think it's really worth, given all
the good brains
we are assembling, to face the most bleeding edge issue of
how to represent
referential phenomena in DS (if it's possible at all),
rather than focusing
on the important but less innovative questions of how to use
evidence from
DS for reference resolution.



\section{Topics and questions}

\paragraph{Reference as a component in DS.}

\paragraph{DS as a component in larger frameworks dealing with reference.}

\paragraph{Reference in the sense of the Pado/Baroni EMNLP paper.}

\paragraph{Commonsense-based coreference.}
Pretty much all Winograd schemas I've seen are examples of
coreference that can be solved based on general commonsense
knowledge, of the sort that distributional semantics should
be good at capturing.

\paragraph{DS models of slots.} Ji and Eisenstein paper.

\paragraph{Maybe we do not need to do anything,
  vectors/embeddings suffice?} Aurelie Herbelot paper.

\paragraph{DS was never meant to be a theory of
  everything. Is reference one of the phenomena that we
  should not use DS to explain?}

\section{Examples}

\paragraph{Plain
pronoun, where
disambiguation really depends on context and not on generic
conceptual
properties}

``The teacher entered the classroom.
He looked scary.''

How do we represent, in distributed terms, ``He looked
scary'' in a way that
captures the fact that it should entail: ``The teacher
looked scary''.

\paragraph{Definite vs indefinite.}

How do we represent, distributionally/distributedly, the
difference between:

The teacher entered the classroom.

and:

A teacher entered the classroom.

How does this representation changes if they are the first
sentences in a
text, or if they are uttered in a context in which we know
that we are
speaking of the classroom where Mr. Smith teaches?

\paragraph{Proper nouns.}

And finally, coming to proper names, how do we capture the
relation between:

A teacher entered the classroom.

and:

John entered the classroom.

(Assuming Mr. Smith's first name is John).

Perhaps, this last issue is just a matter of having enough
training data
for the specific John at hand, but still ``John'' seems to
come with a set of
presupposition that are different from those of ``a
teacher'', and more like
those of ``the teacher'': how do we handle this?

\paragraph{Referentially identical: same or different
  vectors?}
The next and last example is actually from Gemma's MC
project, so if we
decide to use it we should credit it to her:

The Beatles broke up in 1970.

The best-selling band of the sixties broke up in 1970.

Again, how do we capture the relation between these two
sentences?


\bibliography{buecher}
\bibliographystyle{apalike}

\end{document}


